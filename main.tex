\documentclass[stu,12pt,floatsintext]{apa7}

% Language and citation setup
\usepackage[american]{babel}
\usepackage{csquotes}
\usepackage[style=apa,sortcites=true,sorting=nyt,backend=biber]{biblatex}
\DeclareLanguageMapping{american}{american-apa}
\addbibresource{references.bib}

% Font and encoding
\usepackage[T1]{fontenc}
\usepackage{newtxtext,newtxmath}  % Modern Times-like font with math support

% Document metadata
\title{Session 2 Discussion}
\author{Lanie Molinar}
\authorsaffiliations{Colorado Christian University}
\duedate{June 18, 2025,}
\course{Introduction to Systematic Theology (THE-200A)}
\professor{Dr. Cari Nimeth}

\begin{document}

\maketitle
\thispagestyle{plain}
\pagestyle{plain}

\subsection{Introduction}

% This section introduces the scenario and sets the stage for theological reflection.

Imagine you are on a date at your favorite restaurant with someone new. As you are seated at a booth near the back corner, you overhear a child in the next booth praying before eating, saying, "God is great. God is good. And we thank Him for our food." Before you can ask, your date says, "I used to believe what that kid prayed, but I don’t anymore. There has been too much hurt and too much pain to continue to believe that any ‘great’ and ‘good’ God would allow all the evil that is in the world. Either God doesn't care or He is too weak to do anything about it. So I just quit believing." Stunned and saddened by this statement, you pause for a moment, thinking about how best to respond. Your date is clearly struggling with doubt and disillusionment, and you want to offer a thoughtful response. Luckily, you just completed an introductory theology course. This is your chance to put what you learned into practice. The following paragraphs will outline how I would respond to my date's statement, drawing on theological insights from Scripture and doctrine.

\subsection{The Nature of the Problem of Evil}

The problem of evil presents a paradox: if God is all-powerful and wholly good, why does evil exist? This is the classic problem of evil that has challenged theologians and believers throughout history. In this section, I will outline the nature of the problem and how theology provides a framework for understanding it. According to \textcite[chapter 15]{ericksonIntroducingChristianDoctrine2015}, there are two general types of evil: natural evil (suffering caused by natural events like earthquakes or diseases) and moral evil (suffering caused by human actions, such as violence or injustice). Both types of evil raise profound questions about God’s nature and His relationship to the world.

Questions about the problem of evil come in two forms: the religious form, when some aspect of a person's experience makes them question the goodness or greatness of God, and the theological form, which is a more abstract philosophical question about how to reconcile God's attributes with the existence of evil \parencite[chapter 15]{ericksonIntroducingChristianDoctrine2015}. Based on what he said, my date is struggling with the religious form of the problem of evil, questioning how a good and powerful God could allow so much suffering in the world. This is a deeply emotional and intellectual challenge that many believers face at some point in their faith journey. While the problem may not be fully resolved in this life, theology offers meaningful responses that can help address these doubts. In the following section, I will explore several theological themes that help address this tension.

\subsection{Theological Themes for Addressing Evil}

In response to the problem of evil, Christian theology provides several key themes that can help address the doubts expressed by my date. These themes include the necessity of free will, the nature of good and evil, the role of sin, eschatological hope, and the goodness of God. Each of these themes contributes to a comprehensive understanding of how God relates to evil and suffering in the world.

\subsubsection{The Necessity of Free Will}

The first theme is the necessity of free will. According to Christian doctrine, God created humans with the ability to choose between good and evil. This free will is essential for genuine love and obedience. If God forced humans to do good, their actions would not be truly loving or voluntary. Thus, the existence of evil is a necessary consequence of human freedom. As \textcite[chapter 15]{ericksonIntroducingChristianDoctrine2015} explains, the possibility of evil arises from the freedom God granted to humanity. This freedom allows for authentic relationships with God but also opens the door to moral choices that can lead to suffering.

\subsubsection{Reevaluating Good and Evil}

The second theme is the reevaluation of good and evil. What may appear as evil in the short term can serve a greater good in God's broader plan. For example, suffering can lead to perseverance, character development, and hope, as described in \textcite[Romans 5:3-5]{Tyndale1996}. This perspective encourages believers to trust that God is working through all circumstances, even those that seem unjust or painful at the moment. I have seen this in my own life. When I look back at things I have been through, such as health issues, I see that they have shaped me into a stronger person and deepened my faith. This does not mean that all suffering is good in itself, but it can lead to growth and transformation. One of my last attempts to go to college before this was at Penn State, but my health got so bad that I was never able to take one class. At the time, I was devastated when I had to drop out, but now, I am grateful for this, as I have come to realize that the worldview promoted at schools like Penn State, especially around social and political issues, would not have aligned with my faith or convictions. This experience of being redirected away from Penn State, along with a scholarship giveaway by my local Christian music radio station, ultimately led me to Colorado Christian University, where I have found a supportive community and a renewed sense of purpose.

\subsubsection{The Role of Sin and Evil}

The third theme is the role of sin and evil. Christian theology teaches that evil entered the world through humanity's disobedience \parencite[Genesis 3]{Tyndale1996}. This original sin introduced a brokenness into creation that affects all aspects of life. However, not all suffering is a direct punishment for sin; some is simply a result of living in a fallen world. As \textcite[chapter 15]{ericksonIntroducingChristianDoctrine2015} notes, while sin has consequences, it does not mean that every instance of suffering is a divine retribution. This distinction is crucial for understanding how God interacts with human suffering.

\subsubsection{Eschatological Hope}

The fourth theme is eschatological hope. Christian doctrine teaches that evil is temporary and that God will ultimately restore justice and goodness \parencite[Revelation 21:4]{Tyndale1996}. This hope encourages believers to view their suffering in light of eternity, as expressed in \textcite[Romans 8:18]{Tyndale1996}. The promise of a future without pain or sorrow provides comfort and perspective in the midst of current trials. This eschatological hope is not just a distant promise but a present reality that shapes how Christians live in the world today. It reminds us that while we may face suffering now, it is not the end of the story.

Together, these themes offer a foundation for understanding how God remains good and sovereign even in a world marked by suffering. They provide a framework for addressing the doubts expressed by my date, showing that while evil is real and painful, it does not negate God's goodness or power. In the next section, I will explore how these theological insights can be applied to personal experiences of doubt and suffering.

\subsection{The Goodness of God}

In addition to the theological themes discussed above, it is essential to affirm the goodness of God in the face of evil. This affirmation is not merely an abstract doctrine but a core tenet of Christian faith. The goodness of God means that He is inherently loving, just, merciful, and compassionate. As \textcite[Psalm 34:8]{Tyndale1996} says, "Taste and see that the Lord is good. Oh, the joys of those who take refuge in him!" This verse reminds us that God's goodness is not just a theological concept but something that can be experienced, even in hardship.

While there may seem to be tension between God's love and justice, \textcite[chapter 10]{ericksonIntroducingChristianDoctrine2015} emphasizes that God's attributes are not in conflict but rather complement each other. God's justice is loving, and His love is just. This means that while God allows human freedom, He is also actively working to redeem and restore creation. His justice ensures that evil will not go unpunished, while His love provides a way for redemption and healing.

In my own life, I have found that remembering God's goodness, even when I don't understand His reasons, has helped me hold on to hope. Knowing that God is both just and loving gives me confidence that He sees my pain and is using it for a purpose, even when I don't understand His reasons, and as someone living with multiple disabilities and chronic illnesses, I find deep comfort in knowing that He understands and is present with me in my suffering. The fullest expression of God's goodness, however, is found in the person of Jesus Christ, who entered into our suffering and made a way for redemption.

\subsection{Christ as the Ultimate Response to Evil}

Jesus Christ is the ultimate response to the problem of evil. Through His life, death, and resurrection, He demonstrates God's profound love and commitment to humanity. As \textcite[chapter 15]{ericksonIntroducingChristianDoctrine2015} explains, God, in the person of Jesus, did not shy away from suffering but entered into it fully. As Hebrews 4:15 reminds us, “For This High Priest of ours understands our weaknesses, for he faced all of the same testings we do, yet he did not sin.” \parencite{Tyndale1996}. His crucifixion is the clearest evidence of God's willingness to bear the weight of human sin and suffering. His atonement demonstrates both God's love and justice. To satisfy divine justice, Christ bore the punishment for sin Himself, an act that perfectly reveals God's love. This act of sacrificial love shows that God is not distant or indifferent to human suffering but intimately involved in it. Christ's resurrection is the ultimate promise that evil and death do not have the final word. It assures believers that God has triumphed over sin and death, offering hope for a future where suffering will be no more. This truth not only addresses the theological problem of evil but also offers deep personal comfort to those who suffer and question.

\subsection{Personal Response to the Date’s Doubt}

In responding to my date's doubts, I would first acknowledge the pain and confusion behind their statement. It is essential to validate their feelings and recognize that questioning is a natural part of faith. Many believers have wrestled with similar doubts, and it is often through these struggles that faith deepens. Questioning can lead to a deeper understanding of God's nature. I would then share how theology has helped me reconcile God's goodness with the reality of evil. While I do not have all the answers, the themes of free will, the nature of good and evil, the role of sin, eschatological hope, and the goodness of God provide a meaningful framework for understanding suffering. I would emphasize that God is not distant or indifferent but actively involved in our lives, offering comfort and hope in seasons of pain.

Finally, I would invite my date to continue exploring these questions together, perhaps suggesting further reading or discussion. I might say, "I understand why you feel this way, and I want you to know that it is okay to question. Many people have struggled with these same doubts. I have had questions myself, and a couple of years ago, I went through a season where I nearly lost my faith. Through prayer, reading the Bible, and talking with spiritually mature friends, I am thankfully past that now, although I still have questions. I think having questions is a normal part of faith, as long as we do not let them fester into doubt. I believe that God is good and that He is with us in our suffering. What do you think about the idea that God suffers with us?" This approach encourages open dialogue and invites my date to consider a different perspective on their doubts.

\subsection{Conclusion}

The problem of evil is a complex and enduring challenge that Christians have wrestled with throughout history. However, by examining key theological themes such as free will, the nature of good and evil, the role of sin, eschatological hope, and the goodness of God, we can find a framework for understanding suffering. Ultimately, the person of Jesus Christ stands at the center of this discussion, offering hope and redemption in the face of evil. While the problem of evil may not be fully resolved in this life, Christian doctrine provides a foundation for addressing doubts and finding meaning in suffering. As believers, we can hold on to the hope that God is good, that He is present with us in our pain, and that He will ultimately restore all things.

Before doing the reading for this session, I had never considered the possibility that God suffers with us. Now, I find this idea deeply comforting and theologically profound. It reminds me that God is not distant or indifferent to our pain but is intimately involved in our lives, sharing in our struggles and offering hope for the future. What do you think about this idea? I would love to hear your thoughts and continue this conversation, if you are open to it.

\newpage

\printbibliography

\end{document}